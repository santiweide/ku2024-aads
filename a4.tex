\documentclass[12pt]{article}
\usepackage[utf8]{inputenc}
\usepackage{upquote}
\usepackage[margin=1in]{geometry} 
\usepackage{amsmath,amsthm,amssymb}
\usepackage{graphicx}
\usepackage{listings}
\newenvironment{statement}[2][Statement]{\begin{trivlist}
\item[\hskip \labelsep {\bfseries #1}\hskip \labelsep {\bfseries #2.}]}{\end{trivlist}}
\usepackage{xcolor}




\title{Assignment 4}


\author{Author \\
  Wanjing Hu / fng685@alumni.ku.dk  \\
  Zhigao Yan / sxd343@alumni.ku.dk  \\
  Wenshuo Dong / gnj461@alumni.ku.dk  \\
  Jiayi Zhang / xrw579@alumni.ku.dk \\
} 
 

\begin{document}
\maketitle

\section{34.2-10}
%jiayi
\textbf{Question}
Prove that if NP $\neq$ co-NP,  then P $\neq$ NP.\\

\textbf{Answer}
To prove if NP $\neq$ co-NP,  then P $\neq$ NP is equal to prove if P = NP, then NP = co-NP(inverse negation).\\
So assume that P = NP, so for each $L \in P$, $\bar{L} \in P$ ( because we can get the expect result from $\bar{L}$ by reversing the result.)\\
Then from P = NP, we get  $\bar{L} \in NP$ and $L \in NP$.
From the definition of co-NP, $\bar{L} \in NP$ then $L \in co-NP$.\\
Because $L \in co-NP$ and $L \in NP$, we can get NP = co-NP, and the original proposition is valid.


\section{34.3-6}
%wanjing
\textbf{Question}
 A language $L$ is \textit{\textbf{complete}} for a language class $C$ with respect to polynomial-time reductions if $L \in C$ and $L' \le_\text{P} L$ for all $L' \in C$. Show that $\emptyset$ and $\{0, 1\}^{*}$ are the only languages in $\text{P}$ that are not complete for $\text{P}$ with respect to polynomial-time reductions.

\textbf{Answer}
Suppose the polynomial-time reduction is \text{f}:$\{0, 1\}^{*} \rightarrow \{0, 1\}^{*}$. Then $L' \le_\text{P} L$ means for all $x \in \{0,1\}^{*}$

\begin{equation}
x \in L' \; if \, and \; only \; if f(x) \in  L
\end{equation}

$\emptyset$ rejects everything, $\emptyset$  cannot be the $L$.

$\{0, 1\}^{*}$ accepts everything, it is a trivial information. If $x \in L$, x may not $\in L'$, so  $\{0, 1\}^{*}$  cannot be the $L$.

For the rest of the cases, we can construct f as follows:

Let $A_{L'}$ be the deterministic polynomial-time algorithm that decides $L'$. For any input $x \in L'$:

If $A_{L'}(x)=1$, map $x$ to some $f(x) \in L$;

If $A_{L'}(x)=0$, map $x$ to some $f(x) \notin L$;

\textbf{Proof of Correctness}

Since $A_{L'}$  runs in polynomial time, $f(x)$ is computable in polynomial time.

For all $x \in L'$, we have $A_{L'}(x) = 1$, and we have $f(x) \in L$.

For those $x$ whose $f(x) \in L$, assume $A_{L'}(x)=0$, then  $f(x) \notin L$, which is contradict to our assumption. So  $A_{L'}(x)=1$ and $x \in L'$. The symmmetry proof could be made for those $x \notin L'$.

Thus such $f$ exists for any $L' \in P$, $L' \le_\text{P} L$. And $L$ is complete for $P$.

\section{34.4-3}
%zhigao
\textbf{Question: }Professor Jagger proposes to show that \(SAT \leq_P 3-CNF-SAT \) by using only the truth-table technique in the proof of Theorem 34.10, and not the other steps. That is, the professor proposes to take the boolean formula \(\phi\), form a truth table for its variables, derive from the truth table a formula in 3-DNF that is equivalent
to: \(\lnot\phi\), and then negate and apply DeMorgan’s laws to produce a 3-CNF formula
equivalent to \(\phi\). Show that this strategy does not yield a polynomial-time reduction.
\\
\\
This strategy uses only one step, which is the truth table. To build the truth table without using the binary tree, we need \(O(2^n)\) time which n is the number of variables because building the truth table requires enumerating all possible combinations of variable assignments. So it's exponential.

\section{34.4-6}
%wenshuo
\textbf{Question:}\\
Suppose that someone gives you a polynomial-time algorithm to decide formula
satisfiability. Describe how to use this algorithm to find satisfying assignments in
polynomial time.\\
\textbf{Answer:}\\
Given a Boolean formula \( \phi \) in Conjunctive Normal Form (CNF) and a polynomial-time decision procedure for satisfiability (SAT-D), the following steps can be used to find a satisfying assignment in polynomial time:

\begin{enumerate}
    \item \textbf{Input:} A Boolean formula \( \phi \) with variables \( x_1, x_2, \ldots, x_n \).
    \item \textbf{Run the SAT-D algorithm:}
    \begin{itemize}
        \item Evaluate \( \text{SAT-D}(\phi) \). If \( \phi \) is unsatisfiable, output "No satisfying assignment exists" and terminate.
        \item Otherwise, go to the next step.
    \end{itemize}
    \item \textbf{Iterate over each variable:}
    \begin{enumerate}
        \item For each variable \( x_i \):
        \begin{enumerate}
            \item Assign \( x_i = \text{True} \) and create a modified formula \( \phi_{x_i=\text{True}} \) by substituting \( x_i = \text{True} \) in \( \phi \).
            \item Run \( \text{SAT-D}(\phi_{x_i=\text{True}}) \):
            \begin{itemize}
                \item If \( \phi_{x_i=\text{True}} \) is satisfiable, keep \( x_i = \text{True} \).
                \item Otherwise, \( x_i = \text{False} \).
            \end{itemize}
        \end{enumerate}
    \end{enumerate}
    \item \textbf{Output:} After iterating through all variables, the constructed assignment \( (x_1, x_2, \ldots, x_n) \) satisfies \( \phi \).
\end{enumerate}

\section{34.5-1}
%for 34.5-1, the definition of "isomorphic graphs" can be found in Appendix B
%jiayi
\textbf{Question}
The \textbf{subgraph-isomorphism problem} takes two undirected graphs $G_1$ and $G_2$, and it asks whether $G_1$ is isomorphic to a subgraph of $G_2$. Show that the subgraph-isomorphism problem is NP-complete.\\

\textbf{Answer}
From Theorem 34.11, The clique problem is NP-complete. Then we note the subgraph-isomorphism problem as \textbf{SSP} and the clique problem as \textbf{CP}.\\

Firstly, prove SSP $\in$ NP.\\
For given graphs $G_1 = (V_1, E_1)$ and $G_2 = (V_1, E_1)$ ,and a certification like a sub-graph of $G_2$ note as $G_3$, it is easily to check whether $G3$ is isomorphic to $G_1$ in polynomial time, so we get SSP $\in$ NP.\\

Then, prove CP $\in$ SSP.\\
CP: for a given $G = (V, E)$, check whether there is a clique $C$ of size $k$ in $G$.\\
In SSP, if we consider $G_2$ as $G$ in CP, and $G_1$ as $C$ in CP, then the answer of the CP will be the same as the answer of SSP. So we can reduce a SSP to a CP.\\

Since CP is NP-complete, we can get SSP is NP-complete.



\section{34.5-3}
%wanjing
\textbf{Question}
The \textit{integer linear-programming problem} is like the 0-1 integer-programming problem given in Exercise 34.5-2, except that the values of the vector $x$ may be any integers rather than just 0 or 1. Assuming that the 0-1 integer-programming problem is NP-hard, show that the integer linear-programming problem is NP-complete.

(34.5.2)Given an integer $m \times n$ matrix $A$ and an integer $m$-vector $b$, the \textbf{\textit{0-1 integer-programming problem}} asks whether there exists an integer $n$-vector $x$ with elements in the set $\{0, 1\}$ such that $Ax \le b$. Prove that 0-1 integer programming is NP-complete. (Hint: Reduce from 3-CNF-SAT.)

\textbf{Answer}

Here we note the \textit{integer linear-programming problem} as \textbf{ILP}, and the 0-1 integer-programming problem is \textbf{0-1-LP}.

%\textbf{Prove 0-1-LP $\in$ NP}

%\textbf{Prove 0-1-LP $\in$ NPC by proving 3-CNT-SAT $\leq_{P}$ 0-1-LP}

\textbf{Prove ILP $in$ NP}
For a given $x$, we can compute $Ax$ and check if $Ax \leq b$ in polynomial time(more precisely in $O(mn)$). So the verification of ILP is in NP is in polynomial time.

\textbf{Prove ILP $\in$ NPC by proving 0-1-LP $\leq_{P}$ ILP}

Since 0-1-LP is NP-hard, if we reduce ILP to 0-1-LP, ILP will also be NP-hard. 

We introduce additional constraints to the restricted $x$ value to $0 \leq x \leq 1$, which is the same as $x_i \in \{0,1\}$ because $x$ is an integer. Also we take the input $(A,b)$ in the 0-1-LP. Then the LP is the same as 0-1-LP problem.

Since adding constrains to the input x could be done in O(n), which travel the n elements of x and when x is less than 0 then puts 0, when larger than 1 then puts 1, the transformation is in polynomial time.

Thus the reduction is valid and efficient.

From above we can see the ILP is NP-complete.

\section{34.5-6}
%zhigao
\textbf{Question: }Show that the hamiltonian-path problem is NP-complete.\\
\textbf{Proof: } First, we need to prove that hamiltonian-path problem is an NP problem.
Given a graph \(G(V,E)\), the certificate is the sequence of \(|V|\) vertices that makes up the hamiltonian path. To chack the certificate, the algorithm is, firstly check each vertices in \(|V|\) are contained in the sequence. And finally check that there is an edge between each pair of consecutive vertices. So the time complexity is \(O(|V|)\) which means we can verify the certificate in polynomial time. Hence, hamiltonian-path is an NP problem.
\\
\\
Now we need to prove that \(HAM-CYCLE \leq_p HAM-PATH\) since we know the hamiltonian cycle problem is NP-complete. Given a graph \(G(V,E)\) , and construct a graph \(G'(V',E')\). \(V'= V\cup\{s\}\) which means add a new point \(s\) in graph \(G\), \(E' = E \cup\{(s,v)| v\in V\}\) which means add an edge for each point from point \(s\) to each point in \(G\). 
Now, if there is a hamiltonian cycle \(C=v_1\rightarrow v_2\rightarrow\ ... \rightarrow v_n\rightarrow v1\) in graph G, then we can get a hamiltonian-path \(P= s\rightarrow v_2\rightarrow\ ... \rightarrow v_n\) in graph \(G'\). So we prove that hamiltonian cycle problem can be reduced to hamiltonian-path problem and this reduction can be done in polynomial time.



\section{34.5-7}
%wenshuo
\textbf{Question:}\\
The longest-simple-cycle problem is the problem of determining a simple cycle (no repeated vertices) of maximum length in a graph. Formulate a related decision problem, and show that the decision problem is NP-complete.\\
\textbf{Answer:}\\
Longest-Simple-Cycle Decision Problem (LSCD):
\begin{itemize}
    \item \textbf{Instance:} A graph \( G = (V,E) \) (undirected or directed) and a positive integer \( k \leq |V| \).
    \item \textbf{Question:} Does \( G \) contain a simple cycle (a cycle with no repeated vertices) of length at least \( k \)?
\end{itemize}
To prove that LSCD is in NP:
\begin{itemize}
    \item \textbf{Certification:} If the answer is "yes," a certificate is the simple cycle itself, i.e., a sequence of vertices \( (v_1, v_2, \ldots, v_m) \) that forms a cycle with no repeated vertices and \( m \geq k \).
    \item \textbf{Verification:} Given this sequence, we can verify in polynomial time that:
    \begin{enumerate}
        \item Each pair \( (v_i, v_{i+1}) \) (and \( (v_m, v_1) \)) is an edge in \( E \).
        \item All \( v_i \) are distinct.
        \item The length \( m \) is at least \( k \).
    \end{enumerate}
\end{itemize}
Since these checks can be done in polynomial time, LSCD is in NP.\\
To prove that LSCD is NP-hard, we reduce from the Hamiltonian Cycle problem, which is known to be NP-complete.\\
Hamiltonian Cycle Problem:
\begin{itemize}
    \item \textbf{Instance:} A graph \( G' = (V', E') \).
    \item \textbf{Question:} Does \( G' \) contain a simple cycle of length \( |V'| \)? Such a cycle must visit every vertex exactly once.
\end{itemize}
Given an instance \( G' = (V', E') \) of the Hamiltonian Cycle problem, we construct an instance of LSCD as follows:
\begin{itemize}
    \item Let \( G = G' \).
    \item Set \( k = |V'| \).
\end{itemize}
This is a polynomial-time reduction since it involves no complicated transformations just pass the same graph \( G' \) and let \( k = |V'| \).
\begin{itemize}
    \item If \( G' \) has a Hamiltonian cycle, then \( G \) has a simple cycle of length \( |V'| \). Thus, the LSCD instance \( (G, k) \) with \( k = |V'| \) is a "yes" instance.
    \item If \( G' \) does not have a Hamiltonian cycle, then there is no simple cycle visiting all \( |V'| \) vertices in \( G \). Thus, the LSCD instance \( (G, |V'|) \) is a "no" instance.
\end{itemize}
Because the Hamiltonian Cycle problem is a special case of LSCD (obtained by setting \( k = |V| \)), any polynomial-time solution to LSCD would also solve the Hamiltonian Cycle problem. Since the Hamiltonian Cycle problem is NP-complete, LSCD must be NP-hard.\\\\
Therefore, LSCD is NP-complete.



\end{document}
