\documentclass[12pt]{article}
\usepackage[utf8]{inputenc}
\usepackage{upquote}
\usepackage[margin=1in]{geometry} 
\usepackage{amsmath,amsthm,amssymb}
\usepackage{graphicx}
\usepackage{listings}
\newenvironment{statement}[2][Statement]{\begin{trivlist}
\item[\hskip \labelsep {\bfseries #1}\hskip \labelsep {\bfseries #2.}]}{\end{trivlist}}
\usepackage{xcolor}




\title{Assignment 4}


\author{Author \\
  Wanjing Hu / fng685@alumni.ku.dk  \\
  Zhigao Yan / sxd343@alumni.ku.dk  \\
  Wenshuo Dong / gnj461@alumni.ku.dk  \\
  Jiayi Zhang / xrw579@alumni.ku.dk \\
} 
 

\begin{document}
\maketitle

\section{34.2-10}
%jiayi

\section{34.3-6}
%wanjing
\textbf{Question}

\section{34.4-3}
%zhigao
\textbf{Question: }Professor Jagger proposes to show that \(SAT \leq_P 3-CNF-SAT \) by using only the truth-table technique in the proof of Theorem 34.10, and not the other steps. That is, the professor proposes to take the boolean formula \(\phi\), form a truth table for its variables, derive from the truth table a formula in 3-DNF that is equivalent
to: \(\lnot\phi\), and then negate and apply DeMorgan’s laws to produce a 3-CNF formula
equivalent to \(\phi\). Show that this strategy does not yield a polynomial-time reduction.

\section{34.4-6}
%wenshuo

\section{34.5-1}
%for 34.5-1, the definition of "isomorphic graphs" can be found in Appendix B
%jiayi

\section{34.5-3}
%wanjing
\textbf{Question}

\section{34.5-6}
%zhigao
\textbf{Question: }Show that the hamiltonian-path problem is NP-complete.

\section{34.5-7}
%wenshuo

\end{document}
