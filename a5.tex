\documentclass[12pt]{article}
\usepackage[utf8]{inputenc}
\usepackage{upquote}
\usepackage[margin=1in]{geometry} 
\usepackage{amsmath,amsthm,amssymb}
\usepackage{graphicx}
\usepackage{listings}
\newenvironment{statement}[2][Statement]{\begin{trivlist}
\item[\hskip \labelsep {\bfseries #1}\hskip \labelsep {\bfseries #2.}]}{\end{trivlist}}
\usepackage{xcolor}




\title{Assignment 2}


\author{Author \\
  Wanjing Hu / fng685@alumni.ku.dk  \\
  Zhigao Yan / sxd343@alumni.ku.dk  \\
  Wenshuo Dong / gnj461@alumni.ku.dk  \\
  Jiayi Zhang / xrw579@alumni.ku.dk \\
} 
 

\begin{document}
\maketitle

\section{20.3-4}
 %(For the last part, remember that we already use Θ(u) space on the vEBtree. Thus, we can afford to maintain an auxiliary structure using O(u) space without increasing the asymptotic space usage)
%jiayi

\section{20.3-1}
% wanjing
\textbf{Question: } 
Creating a vEB tree with universe size $u$ 

\textbf{Answer:}
We need an extra 64-bit integer representing how many elements are there in the vEB in each entries of its array( naming it the count value). Also for the max and min, we have each an extra  64-bit integer representing how many elements are equals to the max and min. 

The count value changes into 1 whenever the entry value changes into another number. It adds one on every insert operation and the value equals the entry value.

\section{20.3-2}
%zhigao

\section{20.3-3}
%wenshuo

\section{20.3-5}
 %(You may assume k>=2. Your solution should include the dependency on k)
%jiayi

\section{20.3-6}
%wanjing
\textbf{Question: } 
Creating a vEB tree with universe size $u$ requires $O(u)$ time. Suppose we wish to explicitly account for that time. What is the smallest number of operations $n$ for which the amortized time of each operation in a vEB tree is $O(\lg\lg u)$?

\textbf{Answer:}

Performing $n$ operations a vEB tree takes $O(u) + n*O(\lg\lg u)$ time. Using the aggregate amortized analysis, we divide by $n$ to see that the amortized cost of each operations  per operation is:

\begin{equation}
Amortized \, per \, operation = \frac{O(u)}{n} + O(\lg\lg u) 
\end{equation} 

Since the amortized cost is $O(\lg \lg u)$, the initializing operation of $O(u)$ must spread out across  the n operations such that the per-operation contribution of $O(u)/n$ is asymptotically $O(\lg \lg u)$. Its means:

\begin{equation}
\frac{O(u)}{n}=O(\frac{u}{n}) = O( \lg \lg u)
\end{equation}

Thus we need $n \ge u/ \lg \lg u$, which is $\displaystyle \left \lceil \frac{u}{\lg\lg u} \right \rceil$ operations.


\section{Exact and FPT Algorithms - Exercise1}
%zhigao

\section{Exact and FPT Algorithms - Exercise2}
%wenshuo

\section{Exact and FPT Algorithms - Exercise3}
%jiayi

\section{Exact and FPT Algorithms - Exercise4}
%wanjing
%Note: In Exercise 4b the running times we want are $O(m+n+2^k k^2)$ or if you can $O(m+n+2^k)$. You can use the BFP-Kernel(k,g) function from my slides as a subroutine, and may assume you may use without proof that for any $c \in R$ and $f(n) \in O(n^c)$, we have $\sum_{i=0}^{k} 2^{i} f(k-i) \in O(2^k)$.

\textbf{(a)}

For kernalization we remove the neighbours of the vertice whose degree are over k. We travel the vertices and edges of each vertex, costing $O(|V| + |E|)$. 

For identifying the subset $U$, we need to enumerate the subsets $C \in U$ and check whether they are valid vertex cover. After the kerneralization, $|U| = O(k^2)$, so the enumeration costs $O(2^{O(k^2)})$.

Also we have to check if $C$ is a valid vertex cover after each identifying of $C$. We can validate in time $O(|E|)$ by traveling all the edges in that are avaliable after the kernalization. So the cost is $O(|E|)=O(k^2)$.

To sum up the upper bound on the running time is:

\begin{equation}
O(|V| + |E|) + O(2^{O(k^2)}) * O(k^2)
\end{equation}

If we consider the time it takes to identify U and the time it takes to check that a given $C \subseteq U$ is indeed a vertex cover of G \textbf{ONLY}, that will be

\begin{equation}
\begin{aligned}
&O(2^{O(k^2)}) * O(k^2)
&=O^*(2^{O(k^2)}) 
\end{aligned}
\end{equation}


\textbf{(b) - $O(m+n+2^k k^2)$ Solution}

%for any $c \in R$ and $f(n) \in O(n^c)$, we have $\sum_{i=0}^{k} 2^{i} f(k-i) \in O(2^k)$

First we use BFP-Kernel(k,G) to reduce the graph in $O(m+n)$ time, where $m=|V|, n=|E|$. After the kernalization, $|V|=O(k^2)$, $|E|=O(k^2)$.

Then we apply Bounded Search Tree on the kernel. Each branch performs work proportional to $f(k-i)=O(k^2)$ for some $i$ meaning the depth. The $f(k-i)$ means the cost of maintaining or updating the kernel, for example removing vertices or edges, checking constraints.

And the total running time across the branches is:

\begin{equation}
\sum_{i=0}^{k} 2^i f(k-i) = O(2^k k^2)
\end{equation}

So the overall time cost is $O(m+n)+O(2^k k^2) = O(m+n+2^k k^2)$.

\textbf{(b) - $O(m+n+2^k)$ Solution(Not solved...)}

%for any $c \in R$ and $f(n) \in O(n^c)$, we have $\sum_{i=0}^{k} 2^{i} f(k-i) \in O(2^k)$
We try to reduce the kernel along doing the Bounded Search Tree and do the lazy updates.

We mark the vertices or edges as removed in $O(1)$ in the bitmap, when the vertex degree is greater than $k$.

TBD

\end{document}
