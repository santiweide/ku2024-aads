\documentclass[12pt]{article}
\usepackage[utf8]{inputenc}
\usepackage{upquote}
\usepackage[margin=1in]{geometry} 
\usepackage{amsmath,amsthm,amssymb}
\usepackage{graphicx}
\usepackage{listings}
\usepackage{algorithm}
\usepackage[noend]{algpseudocode}
\newenvironment{statement}[2][Statement]{\begin{trivlist}
\item[\hskip \labelsep {\bfseries #1}\hskip \labelsep {\bfseries #2.}]}{\end{trivlist}}
\usepackage{xcolor}




\title{Assignment 2}


\author{Author \\
  Wanjing Hu / fng685@alumni.ku.dk  \\
  Zhigao Yan / sxd343@alumni.ku.dk  \\
  Wenshuo Dong / gnj461@alumni.ku.dk  \\
  Jiayi Zhang / xrw579@alumni.ku.dk \\
} 
 

\begin{document}
\maketitle

\section{20.3-4}
 %(For the last part, remember that we already use Θ(u) space on the vEBtree. Thus, we can afford to maintain an auxiliary structure using O(u) space without increasing the asymptotic space usage)
%jiayi
\textbf{Question: } 
What happens if you call VEB-TREE-INSERT with an element that is already in the vEB tree? What happens if you call VEB-TREE-DELETE with an element that is not in the vEB tree? Explain why the procedures exhibit the behavior that they do. Show how to modify vEB trees and their operations so that we can check in constant time whether an element is present.\\
\textbf{Answer:}
Nothing will happen if users call VEB-TREE-INSERT with an element already in the tree or call VEB-TREE-DELETE with an element not in the tree because vEB tree think all the element is legal by default. When users call a operation with illegal element, it was treat as a normal call and makes a mistake.\\
To assure the legitimacy of the input, we can simply add a hash set to store the elements in the tree and check whether the input is correct before every operation, it will using a extra O(u) space.

\section{20.3-1}
% wanjing
\textbf{Question: } 
Modify vEB trees to support duplicate keys.

\textbf{Answer:}
We need an extra 64-bit integer representing how many elements are there in the vEB in each entries of its array( naming it the count value). Also for the max and min, we have each an extra  64-bit integer representing how many elements are equals to the max and min. 

The count value changes into 1 whenever the entry value changes into another number. It adds one on every insert operation and the value equals the entry value.

\section{20.3-2}
%zhigao
\textbf{Question:} Modify vEB trees to support keys that have associated satellite data.\\
\textbf{Answer:} For any keys, if it is the minimum on some nodes, we need to store the pointer of this key additionally. That is, for non-summary nodes, in addition to storing min also store a pointer(to satellite data) to min since it will not appear after this node.
For a VEB tree of size 2 (only two elements), then also store the pointer to the maximum value. If min = max, the pointer should be associated with the same data. 


For other keys, it is sufficient to simply store the pointer at the location where it appears.
\section{20.3-3}
%wenshuo
\textbf{Question: }Write pseudocode for a procedure that creates an empty van Emde Boas tree.\\
\textbf{Answer: }
\begin{algorithm}
\caption{VEB-CREATE(u)}
\begin{algorithmic}[1]
\Procedure{VEB-CREATE}{u}
    \State $T \gets \text{new VEB\_Tree}$
    \State $T.u \gets u$
    \State $T.\text{min} \gets \text{NIL}$
    \State $T.\text{max} \gets \text{NIL}$
    
    \If{$u > 2$}
        \State $\text{clusterSize} \gets \sqrt{u}$
        \State $T.\text{summary} \gets \text{VEB-CREATE}(\text{clusterSize})$
        \State $T.\text{cluster} \gets \text{new array of size } \text{clusterSize}$
        
        \For{$i \gets 0$ to $\text{clusterSize} - 1$}
            \State $T.\text{cluster}[i] \gets \text{VEB-CREATE}(\text{clusterSize})$
        \EndFor
    \Else
        \State $T.\text{summary} \gets \text{NIL}$
        \State $T.\text{cluster} \gets \text{NIL}$
    \EndIf
    
    \State \Return $T$
\EndProcedure
\end{algorithmic}
\end{algorithm}
\section{20.3-5}
 %(You may assume k>=2. Your solution should include the dependency on k)
%jiayi
\textbf{Question: } 
Suppose that instead of $\sqrt[↑]{\mu}$ clusters, each with universe size $\sqrt[↓]{\mu}$, we constructed vEB trees to have $u^{1/k}$ clusters, each with universe size $u^{1 - 1/k}$, where $k > 1$ is a constant. If we were to modify the operations appropriately, what would be their running times? For the purpose of analysis, assume that $u^{1/k}$ and $u^{1 - 1/k}$ are always integers.\\
\textbf{Answer:}
After constructing vEB tress have  $u^{1/k}$ clusters, each with universe size $u^{1 - 1/k}$, suppose that an operation like insert still has a time complexity $T(\mu)$.\\
So the operation in summary is $T(u^{1/k})$ and in clusters is $T(u^{1 - 1/k})$.\\
The running time would be:
\[
T(\mu) = T(u^{1/k}) + T(u^{1 - 1/k}) + O(1)
\]
When k = 2, it is equal to the original question, so the running time would still be $T(\mu)= O(\lg {\lg \mu})$.\\
When $k > 2$, we can get the $T(\mu)= O(\lg {\lg \mu})$ with similar calculation.
\section{20.3-6}
%wanjing
\textbf{Question: } 
Creating a vEB tree with universe size $u$ requires $O(u)$ time. Suppose we wish to explicitly account for that time. What is the smallest number of operations $n$ for which the amortized time of each operation in a vEB tree is $O(\lg\lg u)$?

\textbf{Answer:}

Performing $n$ operations a vEB tree takes $O(u) + n*O(\lg\lg u)$ time. Using the aggregate amortized analysis, we divide by $n$ to see that the amortized cost of each operations  per operation is:

\begin{equation}
Amortized \, per \, operation = \frac{O(u)}{n} + O(\lg\lg u) 
\end{equation} 

Since the amortized cost is $O(\lg \lg u)$, the initializing operation of $O(u)$ must spread out across  the n operations such that the per-operation contribution of $O(u)/n$ is asymptotically $O(\lg \lg u)$. Its means:

\begin{equation}
\frac{O(u)}{n}=O(\frac{u}{n}) = O( \lg \lg u)
\end{equation}

Thus we need $n \ge u/ \lg \lg u$, which is $\displaystyle \left \lceil \frac{u}{\lg\lg u} \right \rceil$ operations.


\section{Exact and FPT Algorithms - Exercise1}
%zhigao
\textbf{Question:} What is the space needed for the \(O^*(2n)\) time TSP algorithm in Fomin-Kratsch?\\
\textbf{Answer:} The space we need is also exponential since we need to keep all values \(OPT[S,ci]\) for all sets of size \(j\) and \(j+1\). At most we have \(O(2^n)\) for each j, So the space we need is \(O^*(2^n)\)


\section{Exact and FPT Algorithms - Exercise2}
%wenshuo
\textbf{Answer A: }\\
We aim to prove:
\[
T(n) \leq 2 \times 3^{n/3} - 1
\]
for \( n \geq 2 \) when \( s = 1 \). The recursive relation is:
\[
T(n) \leq 1 + T(n-1)
\]
and the inductive hypothesis assumes:
\[
T(n-1) \leq 2 \times 3^{(n-1)/3} - 1.
\]

\subsubsection*{Step 1: Substitute the inductive hypothesis}
Using the recursive relation, we substitute \( T(n-1) \) with the inductive hypothesis:
\[
T(n) \leq 1 + \left( 2 \times 3^{(n-1)/3} - 1 \right).
\]

\subsubsection*{Step 2: Simplify the expression}
Simplify the terms:
\[
T(n) \leq 1 + 2 \times 3^{(n-1)/3} - 1 = 2 \times 3^{(n-1)/3}.
\]

\subsubsection*{Step 3: Rewrite \( 3^{(n-1)/3} \) in terms of \( 3^{n/3} \)}
Using the property \( 3^{(n-1)/3} = 3^{n/3} \cdot 3^{-1/3} \), rewrite the expression:
\[
T(n) \leq 2 \times 3^{n/3} \cdot 3^{-1/3}.
\]

\subsubsection*{Step 4: Prove the inequality}
To prove \( T(n) \leq 2 \times 3^{n/3} - 1 \), it suffices to show:
\[
2 \times 3^{n/3} \cdot 3^{-1/3} \leq 2 \times 3^{n/3} - 1.
\]

Divide through by \( 2 \times 3^{n/3} \) (valid for \( n \geq 2 \)):
\[
3^{-1/3} \leq 1 - \frac{1}{2 \times 3^{n/3}}.
\]

For \( n \geq 2 \), \( 3^{n/3} \geq 3^{2/3} \approx 1.442 \), so:
\[
\frac{1}{2 \times 3^{n/3}} \leq \frac{1}{2 \times 1.442} \approx 0.346.
\]

Thus:
\[
1 - \frac{1}{2 \times 3^{n/3}} \geq 1 - 0.346 = 0.654.
\]

Meanwhile, \( 3^{-1/3} \approx 0.693 \), which satisfies:
\[
3^{-1/3} \leq 0.654.
\]

\subsubsection*{Conclusion}
For \( n \geq 2 \), the inequality holds:
\[
T(n) \leq 2 \times 3^{n/3} - 1.
\]
\textbf{Answer B: }\\
We aim to complete the inductive step for \( n \geq 2 \) and \( s > 1 \), using the given lemma:
\[
\left( \frac{s}{3} \right)^s \leq 1 \quad \text{for all integers } s.
\]

\subsubsection*{Step 1: Restating the inequality}
The recursive relation is:
\[
T(n) \leq 1 + s \cdot T(n-s).
\]
Using the inductive hypothesis:
\[
T(n-s) \leq 2 \cdot 3^{(n-s)/3} - 1,
\]
we substitute this into the relation:
\[
T(n) \leq 1 + s \cdot (2 \cdot 3^{(n-s)/3} - 1).
\]

Expanding:
\[
T(n) \leq 1 + 2s \cdot 3^{(n-s)/3} - s.
\]

\subsubsection*{Step 2: Rewrite \( 3^{(n-s)/3} \) in terms of \( 3^{n/3} \)}
Express \( 3^{(n-s)/3} \) as \( 3^{n/3} \cdot 3^{-s/3} \):
\[
T(n) \leq 1 + 2s \cdot 3^{n/3} \cdot 3^{-s/3} - s.
\]

To bound \( T(n) \), we need to show:
\[
1 + 2s \cdot 3^{n/3} \cdot 3^{-s/3} - s \leq 2 \cdot 3^{n/3} - 1.
\]

Rearranging terms:
\[
2s \cdot 3^{n/3} \cdot 3^{-s/3} \leq 2 \cdot 3^{n/3} - s - 2.
\]

\subsubsection*{Step 3: Simplify using the lemma}
The key component is \( s \cdot 3^{-s/3} \), which is bounded by Lemma 1:
\[
\left( \frac{s}{3} \right)^s \leq 1.
\]

Taking the \( s \)-th root:
\[
s \cdot 3^{-s/3} \leq 1.
\]

Thus:
\[
2s \cdot 3^{n/3} \cdot 3^{-s/3} \leq 2 \cdot 3^{n/3}.
\]

\subsubsection*{Step 4: Complete the inequality}
With \( s \cdot 3^{-s/3} \leq 1 \), the inequality simplifies:
\[
1 + 2 \cdot 3^{n/3} - s \leq 2 \cdot 3^{n/3} - 1.
\]

For \( n \geq 2 \), it is evident \( s \geq 1 \), ensuring the inequality holds:
\[
T(n) \leq 2 \cdot 3^{n/3} - 1.
\]

\subsubsection*{Conclusion}
Using the inductive hypothesis and Lemma 1, we have proven \( T(n) \leq 2 \cdot 3^{n/3} - 1 \) for \( n \geq 2 \) and \( s > 1 \).
\
\section{Exact and FPT Algorithms - Exercise3}
%jiayi
\textbf{Question:}
What is the space needed for the above recursive independent set algorithm from Fomin-Kratsch?\\
\textbf{Answer:} \\
The algorithm is a recursive algorithm and it will not store the whole tree in each space in the recursive stack, it only need to store the local information in each step of recursion.\\
So the space needed only related to the depth of recursion, in Fomin-Kratsch, the depth will less or equal than the input size n, then the space complexity is O(n).

\section{Exact and FPT Algorithms - Exercise4}
%wanjing
%Note: In Exercise 4b the running times we want are $O(m+n+2^k k^2)$ or if you can $O(m+n+2^k)$. You can use the BFP-Kernel(k,g) function from my slides as a subroutine, and may assume you may use without proof that for any $c \in R$ and $f(n) \in O(n^c)$, we have $\sum_{i=0}^{k} 2^{i} f(k-i) \in O(2^k)$.

\textbf{(a)}

For kernalization we remove the neighbours of the vertice whose degree are over k. We travel the vertices and edges of each vertex, costing $O(|V| + |E|)$. 

For identifying the subset $U$, we need to enumerate the subsets $C \in U$ and check whether they are valid vertex cover. After the kerneralization, $|U| = O(k^2)$, so the enumeration costs $O(2^{O(k^2)})$.

Also we have to check if $C$ is a valid vertex cover after each identifying of $C$. We can validate in time $O(|E|)$ by traveling all the edges in that are avaliable after the kernalization. So the cost is $O(|E|)=O(k^2)$.

To sum up the upper bound on the running time is:

\begin{equation}
O(|V| + |E|) + O(2^{O(k^2)}) * O(k^2)
\end{equation}

If we consider the time it takes to identify U and the time it takes to check that a given $C \subseteq U$ is indeed a vertex cover of G \textbf{ONLY}, that will be

\begin{equation}
\begin{aligned}
&O(2^{O(k^2)}) * O(k^2)
&=O^*(2^{O(k^2)}) 
\end{aligned}
\end{equation}


\textbf{(b) - $O(m+n+2^k k^2)$ Solution}
%wanjing
%for any $c \in R$ and $f(n) \in O(n^c)$, we have $\sum_{i=0}^{k} 2^{i} f(k-i) \in O(2^k)$

First we use BFP-Kernel(k,G) to reduce the graph in $O(m+n)$ time, where $m=|V|, n=|E|$. After the kernalization, $|V|=O(k^2)$, $|E|=O(k^2)$.

Then we apply Bounded Search Tree on the kernel. Each branch performs work proportional to $f(k-i)=O(k^2)$ for some $i$ meaning the depth. The $f(k-i)$ means the cost of maintaining or updating the kernel, for example removing vertices or edges, checking constraints.

And the total running time across the branches is:

\begin{equation}
\sum_{i=0}^{k} 2^i f(k-i) = O(2^k k^2)
\end{equation}

So the overall time cost is $O(m+n)+O(2^k k^2) = O(m+n+2^k k^2)$.

\textbf{(b) - $O(m+n+2^k)$ Solution(Not solved...)}

%for any $c \in R$ and $f(n) \in O(n^c)$, we have $\sum_{i=0}^{k} 2^{i} f(k-i) \in O(2^k)$
We try to reduce the kernel along doing the Bounded Search Tree and do the lazy updates.

We mark the vertices or edges as removed in $O(1)$ in the bitmap, when the vertex degree is greater than $k$.

TBD

\end{document}
