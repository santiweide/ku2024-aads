\documentclass[12pt]{article}
\usepackage[utf8]{inputenc}
\usepackage{upquote}
\usepackage[margin=1in]{geometry} 
\usepackage{amsmath,amsthm,amssymb}
\usepackage{graphicx}
\usepackage{listings}
\newenvironment{statement}[2][Statement]{\begin{trivlist}
\item[\hskip \labelsep {\bfseries #1}\hskip \labelsep {\bfseries #2.}]}{\end{trivlist}}
\usepackage{xcolor}




\title{Assignment 3}


\author{Author \\
  Wanjing Hu / fng685@alumni.ku.dk  \\
  Zhigao Yan / sxd343@alumni.ku.dk  \\
  Wenshuo Dong / gnj461@alumni.ku.dk  \\
  Jiayi Zhang / xrw579@alumni.ku.dk \\
} 
 

\begin{document}
\maketitle
% The ones marked * are the hard, and will get larger weight.

\section{Hashing-2.1}
\textbf{Question: } Is the truly independent hash function h : U $\rightarrow$ [m] universal?
%wanjing

Yes. 

Our reason is that for truly independent hash function, for any input $x_i \in U$, $i = 1,2,...,n$, the outputs $h(x_i)$ are completely independent and uniformly distributed over $[m]={0,1,...,m-1}$. Assume we are picking the value $k$ such that $h(x)=k$ , then we have

\begin{equation}
\begin{aligned}
\mathop{Pr} \limits_{h} [h(x)=k] \leq \frac{1}{m}\\
\mathop{Pr} \limits_{h} [h(y)=k] \leq \frac{1}{m}\\
\end{aligned}
\end{equation}
That is $Pr[h(x)=h(y)] \leq \frac{1}{m}$. So the truly independent hash function h : U $\rightarrow$ [m] is universal.

\section{Hashing-2.2}
%zhigao

\section{Hashing-2.3}
%wenshuo

\section{Hashing-2.4 }
% in (b) assume x is not in S
%jiayi

\section{Hashing-2.5*}
%wanjing
\textbf{Question: }
With $s: U \rightarrow [n^3]$ and $h : U \rightarrow [n]$ independent universal hash functions, for a given $x \in U \backslash S$, what is the probability of a false positive when we search $x$, that is, what is the probability that there is a key $y \in S$ such that $h(y) = h(x)$ and $s(y) = s(x)$?

\textbf{Answer}

Now we have:
\begin{equation}
\begin{aligned}
\mathop{Pr} \limits_{h} [h(x)=h(y)] &= \frac{1}{n}\\
\mathop{Pr} \limits_{s} [s(x)=s(y)] &= \frac{1}{n^3}
\end{aligned}
\end{equation}

And we want to know about:
\begin{equation}
\begin{aligned}
&\mathop{Pr}  [False \, Positive]\\
&= Pr[\exists y \in S \, such \, that \, h(y) = h(x) \, and \, s(y) = s(x)]\\
&<= \sum_{y \in S} Pr[h(x)=h(y) \, and \, s(y)=s(x)]
\end{aligned}
\end{equation}

The inequality is a union bound, that the probability of that at least one of multiple events happen is at most the sum of their probabilities(reference to the hashing material provided in class). Since h and s are independent, for any single $y \in S$ we have:
\begin{equation}
\begin{aligned}
&\mathop{Pr} [h(x)=h(y) \, and \, s(y)=s(x)] \\
&\mathop{Pr} [h(x)=h(y)] * \mathop{Pr} \limits_{h,s} [s(x)=s(y)] \\
&= \frac{1}{n^4}
\end{aligned}
\end{equation}

So we have
\begin{equation}
\begin{aligned}
&\mathop{Pr}  [False \, Positive]\\
&<= |S|/n^4
\end{aligned}
\end{equation}

Where $|S|$ is the size of set S.

\section{Hashing-2.6*}
%zhigao

\section{Hashing-3.1}
%wenshuo

\section{Hashing-3.2}
%jiayi

\section{Hashing-3.3* }
% you must prove that the collision probability is at most c/m. It is not sufficient to prove c²/m
%wanjing
\textbf{Question: }
Argue that if $h: U \rightarrow [m]$ is c-approximately strong universal, then h is also c-approximately universal.

\textbf{Answer}
The collision probability of the c-approximately strong universal is 

\begin{equation}
\begin{aligned}
&Pr[h(x)=h(y)] \\
&= \sum_{q \in [m]} Pr[h(x) = q \, and \, h(y) = q]\\
&= \sum_{q \in [m]} Pr[h(x) = q \, and \, h(y) = h(x)]\\
&=\sum_{q \in [m]} Pr[h(x) = q] * Pr[h(y) = h(x)]\\
&=\sum_{q \in [m]} Pr[h(x) = q] *1/m
\end{aligned}
\end{equation}

For the last equation we use the independency property of c-approximately universal. And we also have:
\begin{equation}
\begin{aligned}
Pr[h(x) = q] &<= \frac{c}{m} \\
\end{aligned}
\end{equation}

So we have
\begin{equation}
\begin{aligned}
&Pr[h(x)=h(y)] \\
&= \sum_{q \in [m]} Pr[h(x) = q] * 1/m\\
&<=\sum_{q \in [m]}   \frac{c}{m^2}
= \frac{c}{m}
\end{aligned}
\end{equation}

\section{Hashing-3.4*}
% hint: the answer is no, because not all pairs of keys hash independently - your job is to prove this.
%zhigao

\section{Hashing-3.5}
%wenshuo

\section{Hashing-3.6}
%jiayi

\section{Complexity-34.1-1}
%wanjing
\textbf{Question}
Define the optimization problem $LONGEST-PATH-LENGTH$ as the relation that associates each instance of an undirected graph and two vertices with the number of edges in a longest simple path between the two vertices. Define the decision problem $LONGEST-PATH= \{\langle G, u, v, k\rangle: G = (V, E)$ is an undirected graph, $u, v \in V, k \ge 0$ is an integer, and there exists a simple path from $u$ to $v$ in $G$ consisting of at least $k$ edges $\}$. Show that the optimization problem $LONGEST-PATH-LENGTH$ can be solved in polynomial time if and only if $LONGEST-PATH \in P$.


\section{Complexity-34.1-5}
%zhigao

\section{Complexity-34.2-5}
%wenshuo

\section{Complexity-34.2-6}
%jiayi

\section{Complexity-34.2-8}
%wanjing
\textbf{Question}
Let $\phi$ be a boolean formula constructed from the boolean input variables $x_1, x_2, \dots, x_k$, negations ($\neg$), ANDs ($\vee$), ORs ($\wedge$), and parentheses. The formula $\phi$ is a \textit{\textbf{tautology}} if it evaluates to $1$ for every assignment of $1$ and $0$ to the input variables. Define $TAUTOLOGY$ as the language of boolean formulas that are tautologies. Show that $TAUTOLOGY \in co-NP$.


\section{Complexity-34.3-2}
%zhigao

\section{Complexity-34.3-3}
%wenshuo

\end{document}
