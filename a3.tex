\documentclass[12pt]{article}
\usepackage[utf8]{inputenc}
\usepackage{upquote}
\usepackage[margin=1in]{geometry} 
\usepackage{amsmath,amsthm,amssymb}
\usepackage{graphicx}
\usepackage{listings}
\newenvironment{statement}[2][Statement]{\begin{trivlist}
\item[\hskip \labelsep {\bfseries #1}\hskip \labelsep {\bfseries #2.}]}{\end{trivlist}}
\usepackage{xcolor}




\title{Assignment 3}


\author{Author \\
  Wanjing Hu / fng685@alumni.ku.dk  \\
  Zhigao Yan / sxd343@alumni.ku.dk  \\
  Wenshuo Dong / gnj461@alumni.ku.dk  \\
  Jiayi Zhang / xrw579@alumni.ku.dk \\
} 
 

\begin{document}
\maketitle
% The ones marked * are the hard, and will get larger weight.

\section{Hashing-2.1}
%wanjing

\section{Hashing-2.2}
%zhigao

\section{Hashing-2.3}
%wenshuo

\section{Hashing-2.4 }
% in (b) assume x is not in S
%jiayi

\section{Hashing-2.5*}
%wanjing

\section{Hashing-2.6*}
%zhigao

\section{Hashing-3.1}
%wenshuo

\section{Hashing-3.2}
%jiayi

\section{Hashing-3.3* }
% you must prove that the collision probability is at most c/m. It is not sufficient to prove c²/m
%wanjing

\section{Hashing-3.4*}
% hint: the answer is no, because not all pairs of keys hash independently - your job is to prove this.
%zhigao

\section{Hashing-3.5}
%wenshuo

\section{Hashing-3.6}
%jiayi

\section{Complexity-34.1-1}
%wanjing

\section{Complexity-34.1-5}
%zhigao

\section{Complexity-34.2-5}
%wenshuo

\section{Complexity-34.2-6}
%jiayi

\section{Complexity-34.2-8}
%wanjing

\section{Complexity-34.3-2}
%zhigao

\section{Complexity-34.3-3}
%wenshuo

\end{document}
