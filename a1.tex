\documentclass[12pt]{article}
\usepackage[utf8]{inputenc}
\usepackage{upquote}
\usepackage[margin=1in]{geometry} 
\usepackage{amsmath,amsthm,amssymb}
\usepackage{graphicx}
\usepackage{listings}
\newenvironment{statement}[2][Statement]{\begin{trivlist}
\item[\hskip \labelsep {\bfseries #1}\hskip \labelsep {\bfseries #2.}]}{\end{trivlist}}
\usepackage{xcolor}




\title{Assignment 1}


\author{Author \\
  Wanjing Hu / fng685@alumni.ku.dk  \\
  Zhigao Yan / sxd343@alumni.ku.dk  \\
  Wanjing Hu / fng685@alumni.ku.dk  \\
  Wanjing Hu / fng685@alumni.ku.dk  \\
} 
 

\begin{document}
\maketitle


\section{26.1-1}
%wanjing
\textbf{Question: }Show that splitting an edge in a flow network yields an equivalent network. More formally, suppose that flow network $G$ contains edge $(u, v)$, and we create a new flow network $G'$ by creating a new vertex $x$ and replacing $(u, v)$ by new edges $(u, x)$ and $(x, v)$ with $c(u, x) = c(x, v) = c(u, v)$. Show that a maximum flow in $G'$ has the same value as a maximum flow in $G$.

\section{26.1-4}
%wanjing

\section{26.1-7}

\section{26.2-2}
\section{26.2-4}
%zhigao
\begin{figure}[h]
    \centering
    \includegraphics[width=0.5\linewidth]{截屏2024-11-18 下午8.43.56.png}
    \caption{26.1(a)}
    \label{fig:26.1(a)}
\end{figure}
By executing the Edmonds-Karp algorithm in 26.1(a), the first shortest path found is 
\[s \rightarrow v_1 \rightarrow v_3 \rightarrow t\]
The maximum flow on this path is 12.
Then updated the residual network:
\[s \rightarrow v1: 4, \quad v1 \rightarrow v3: 0, \quad v3 \rightarrow t: 8\]
\[v1 \rightarrow s: 12, \quad v3 \rightarrow v1: 12, \quad t \rightarrow v3: 12\]
Then use the updated residual network to find the next path:
\[s  \rightarrow v_2 \rightarrow v_4 \rightarrow t\]
The maximum flow on this path is 4.
Then updated the residual network:
\[s \rightarrow v2: 9, \quad v2 \rightarrow v4: 10, \quad v4 \rightarrow t: 0 \]
\[v2 \rightarrow s: 4, \quad v4 \rightarrow v2: 4, \quad t \rightarrow v4: 4 \]
Repeatedly, use the updated residual network to find the next path:
\[s \rightarrow v_2 \rightarrow v_4  \rightarrow v_3 \rightarrow t\]
The maximum flow on this path is 7.
Then updated the residual network:
\[s \rightarrow v2: 2, \quad v2 \rightarrow v4: 3, \quad v4 \rightarrow v3: 0 \quad v3 \rightarrow t: 1\]
\[v2 \rightarrow s: 11, \quad v4 \rightarrow v2: 11, \quad v3 \rightarrow v4: 7 \quad t \rightarrow v3: 19\]
At this point no other path can be found, so the total maximum flow is \(12+4+7=23\).
\section{26.2-3}
\begin{figure}[h]
    \centering
    \includegraphics[width=0.5\linewidth]{截屏2024-11-18 下午8.46.37.png}
    \caption{26.1(b)}
    \label{fig:enter-label}

\end{figure}
Calculated in accordance with CLRS formulas 26.9 and 26.10.

\[f(S, T) = \sum_{u \in S} \sum_{v \in T} f(u, v) - \sum_{u \in S} \sum_{v \in T} f(v, u)\]
\[c(S, T) = \sum_{u \in S} \sum_{v \in T} c(u, v)\]

\[f(S, T) = f(s,v1)+f(v2,v1)+f(v4,v3)+f(v4,t) - f(v3,v2) = 11+1+7+4-4=19\]
\[c(S, T) = c(s,v1)+c(v2,v1)+c(v4,v3)+c(v4,t)=16+4+7+4=31\]
%zhigao
\section{26.2-7}
Firstly, I need to check the capacity constraint.\\
We know that : \(c_f(p) = \min \{ c_f(u, v) : (u, v) \text{ is on } p \}\)
Hence we have the following:
\[c_f(p) \leq c_f(u, v) \quad \forall (u, v) \in p.\]
So the capacity constraint is satisfied.\\
Then check the flow conservation.\\
We know the \(p\) is an augmenting path, And from the definition:
\[f_p(u, v) =
\begin{cases} 
c_f(p), & \text{if } (u, v) \text{ is on } p, \\ 
0, & \text{otherwise}.
\end{cases}\]
Assume that \(w\) is an output point and that \(w \in p\), Hence:
\[f_p(v, w) = c_f(p)\]
So we get: 
\[\sum_{u \in p} f_p(u, v) =\sum_{w \in p} f_p(v, w)\]
The flow conservation is satisfied.\\
By the definition of the augmenting path, the maximum amount by which we can increase the flow on each edge in an augmenting path is the residual capacity of \(p\).
So the flow on each edge is \(c_f(p)\), so \( |f_p| = c_f(p)> 0\)
%zhigao
\section{26.2-9}
\section{26.3-2}

\end{document}
