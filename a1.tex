\documentclass[12pt]{article}
\usepackage[utf8]{inputenc}
\usepackage{upquote}
\usepackage[margin=1in]{geometry} 
\usepackage{amsmath,amsthm,amssymb}
\usepackage{graphicx}
\usepackage{listings}
\newenvironment{statement}[2][Statement]{\begin{trivlist}
\item[\hskip \labelsep {\bfseries #1}\hskip \labelsep {\bfseries #2.}]}{\end{trivlist}}
\usepackage{xcolor}




\title{Assignment 1}


\author{Author \\
  Wanjing Hu / fng685@alumni.ku.dk  \\
  Zhigao Yan / sxd343@alumni.ku.dk  \\
  Wenshuo Dong / gnj461@alumni.ku.dk  \\
  Jiayi Zhang / xrw579@alumni.ku.dk \\
} 
 

\begin{document}
\maketitle


\section{26.1-1}
%wanjing
\textbf{Question: }Show that splitting an edge in a flow network yields an equivalent network. More formally, suppose that flow network $G$ contains edge $(u, v)$, and we create a new flow network $G'$ by creating a new vertex $x$ and replacing $(u, v)$ by new edges $(u, x)$ and $(x, v)$ with $c(u, x) = c(x, v) = c(u, v)$. Show that a maximum flow in $G'$ has the same value as a maximum flow in $G$.

\textbf{Answer: }As noted in the Question, we have $G'=(V',E')$ where $V'=V\cup \{x\}$, $E'=E-\{(u,v)\}\cup\{(u,x), (v, x)\}$.

First we should have a flow in $G'$ that is both a flow and a maximum flow. Construct $f'$ in $G'$ as follows:
\begin{equation}
f'(y,z) = 
\begin{cases}
f(y,z) &\mbox{if (y,z) $\neq$ (u,x) and (y,z) $\neq$ (x,v)}\\
f(u,v) &\mbox{if (y,z) = (u,x) or (y,z) = (x,v)}.
\end{cases}
\end{equation}
, where the $f$ is the maximum flow in $G$, and the only difference between $f$ and $f'$ is the $f(u,v)$ is replaced by two flows on the new added vertex $x$, valuing $f(u,v)$.

\subsection{Proof of flow - capability constraints}

According to the $c(u, x) = c(x, v) = c(u, v)$ in Question, we have:

\begin{gather}
f'(u,x) = f(u,v) <= c(u,v) = c(u,x)\\ \nonumber
f'(x,v) = f(u,v)<=c(u,v) = c(x,v)\\ \nonumber
f'(y,z) = f(y,z) <= c(y,z), when (y,z) \neq (u,x) and (y,z) \neq (x,v) \nonumber
\end{gather}

Also we have the $f'(y,z)>= 0$ for any $(y,z)$.

So the $f'$ we constructed obeys the flow capability constraints.

\subsection{Proof of flow - flow conservation}

Here we look at the flow going throw three vertex: $u$, $v$, and $x$. The rest of the vertex are naturally proved by $f$ is obeying the flow conservation. 

\textbf{Case $u$} Assume $u \notin \{s,t\} $, since $f$ obeys flow conservation, we can do as follows:

\begin{equation}
\begin{aligned}
\sum_{y \in V'} f'(u,y) &= \sum_{y \in V - \{x\}} f'(u,y) + f'(u,x)\\
&= \sum_{y \in V - \{v\}} f(u,y) + f(u,v)\\
&= \sum_{y \in V} f(u,y) \\
&= \sum_{y \in V} f(y,u) \\
&= \sum_{y \in V} f'(y,u) \\
\end{aligned}
\end{equation}

\textbf{Case $v$}
This is quite symmetric to the case $u$.
\begin{equation}
\begin{aligned}
\sum_{y \in V'} f'(y,v) &= \sum_{y \in V - \{x\}} f'(y,v) + f'(x,v)\\
&= \sum_{y \in V - \{u\}} f(y,v) + f(u,v)\\
&= \sum_{y \in V} f(y,v) \\
&= \sum_{y \in V} f(v,y) \\
&= \sum_{y \in V} f'(v,y) \\
\end{aligned}
\end{equation}

\textbf{Case $x$}

\begin{equation}
\begin{aligned}
\sum_{y \in V'} f'(y,x) &= f'(u,x)
&= f(u,v)
&= f'(x,v)
&= \sum_{y \in V'} f'(x,y)
\end{aligned}
\end{equation}


So the $f'$ we constructed obeys the flow flow conservation, and $f'$ is a valid flow.

\subsection{Proof of maximum flow}
Here we have two cases: one is $s \in \{u, v\}$ and the other is $s \notin \{u, v\}$. 

For the first case, we here prove the $s=u$ case since the $s=v$ case would be symmetric.
\begin{equation}
\begin{aligned}
 |f'| &= f_s\\
 &= \sum_{y \in V'} f'(u,y) - \sum_{y \in V'} f'(y,u) \\
 &= \sum_{y \in V'-{x}} f'(u,y) + f'(u,x) - \sum_{y \in V'} f'(y,u)\\
 &= \sum_{y \in V-{v}} f(u,y) + f(u,v) - \sum_{y \in V} f(y,u)\\
 &= \sum_{y \in V} f(u,y)  - \sum_{y \in V} f(y,u)\\
 &=|f|
\end{aligned}
 \end{equation}

For the second case, we have $x$ not connected to $s$ or $t$. Then we can prove as follows:
\begin{equation}
\begin{aligned}
 |f'| &= f_s\\
 &= \sum_{y \in V'} f'(s,y) - \sum_{y \in V'} f'(y,s) \\
 &= \sum_{y \in V-{x}} f'(s,y) + f'(s,x) - \sum_{y \in V-{x}} f'(y,s) - f'(x,s) \\
 &= \sum_{y \in V} f'(s,y) + 0 - \sum_{y \in V} f'(y,s) - 0 \\
 &= \sum_{y \in V} f(s,y) - \sum_{y \in V} f(y,s)  \\
 &= |f|
 \end{aligned}
 \end{equation}

As all above, the proof is done. We also notice that we did not make use of the feature that $f$ is a maximum flow in $G$, so for all the flows the $|f'| =|f|$ hold.

%%% TODO
%You only proved that for each f you can construct such f', which means maxflow(G')>=maxflow(G). What about the other direction?


$Q.E.D.$
\section{26.1-4}
%wanjing
\textbf{Question: } Let $f$ be a flow network, and let $a$ be a real number. The scalar flow product denoted $af$, is a function from V x V to $\mathbb{R}$ defined by 

$(af)(u, v) = a \cdot f(u, v).$

Prove that the flows in a network form a convex set. That is, show that if $f_1$ and $f_2$ are flows, then so is $af_1+(1-a)f_2$ for all $a$ in the range $0<=a<=1$

\textbf{Answer:}

We are asked to prove that for all $0<=a<=1$, $af_1+(1-a)f_2$ satisfy both the flow conservation and the capacity constraint when $f_1$ and $f_2$ are valid flows. Let $G=(V,E)$ with $\{s, t\}$ and capability $c$, and $f1$, $f2$ are two flows in $G$.


\subsection{Proof of capability constraint}
First we prove the upper bound of the capability constraint:
\begin{equation}
\begin{aligned}
f'(u,v) &= (af_1)(u,v)+((1-a)f_2)(u,v) \\
 &= a \cdot f_1(u,v)+(1-a) \cdot  f_2(u,v) \\
 &<= a \cdot c(u,v)+(1-a) \cdot  c(u,v) \\
 &= c(u,v) \\
\end{aligned}
\end{equation}

Second we prove the lower bound of the capability constraint:
\begin{equation}
\begin{aligned}
f'(u,v) &= (af_1)(u,v)+((1-a)f_2)(u,v) \\
 &>= a \cdot 0+(1-a) \cdot 0 \\
 &= 0 \\
\end{aligned}
\end{equation}


\subsection{Proof of flow conservation}

For $f1$ and $f2$ we have;
\begin{equation}
\begin{aligned}
\sum_{y \in V} f_1(u,y) - \sum_{y \in V} f_1(y,u) &= 0 \\
\sum_{y \in V} f_2(u,y) - \sum_{y \in V} f_2(y,u) &= 0
\end{aligned}
\end{equation}


For $u \in V$, $u \notin \{s, t\}$: 
\begin{equation}
\begin{aligned}
&\sum_{y \in V} f'(u,y) - \sum_{y \in V} f'(y,u) \\
&= \sum_{y \in V} a \cdot f_1(u,y) + (1-a) \cdot f_2(u,y)  - \sum_{y \in V}  a \cdot f_1(y,u) + (1-a) \cdot f_2(y,u)\\ 
&= \sum_{y \in V} a \cdot f_1(u,y) + \sum_{y \in V}(1-a) \cdot f_2(u,y) - \sum_{y \in V}  a \cdot f_1(y,u) -  \sum_{y \in V}  (1-a) \cdot f_2(y,u)\\
&= \sum_{y \in V} a \cdot f_1(u,y)  - \sum_{y \in V}  a \cdot f_1(y,u) + \sum_{y \in V}(1-a) \cdot f_2(u,y) -  \sum_{y \in V}  (1-a) \cdot f_2(y,u)\\
&= 0 + 0\\
&= 0
\end{aligned}
\end{equation}

$Q.E.D.$

\section{26.1-7}

%jiayi
\textbf{Question:} Suppose that, in addition to edge capacities, a flow network has \textbf{vertex capacities}. That is, each vertex $v$ has a limit $l(v)$ on how much flow can pass through $v$. Show how to transform a flow network $G = (V, E) $ with vertex capacities into an equivalent flow network $G = (V, E) $ without vertex capacities, such that a maximum flow in $G'$ has the same value as a maximum flow in $G$. How many vertices and edges does  $G'$ have?

\textbf{Answer:} 
For every non-source and non-sink vertex $v$ in G, add a new vertex $v'$, and the original vertex $v$ will succeed all the edges $(u, v)$ from the other vertexes $u$ to $v$ with the capacity $c(u, v)$, as $v'$ get the edges from $v$ to the other vertexes $(v, u)$ with the capacity  $c(v, u)$.
Then, add a new edges  $(v, v')$ with a capacity $c(v, v')$ = $l(v)$.

So, the amount of the vertexes (except the source $s$ and the sink $t$) of the graph will be doubled, and one new edge will be added for each vertexes pairs. 
Suppose that, there are $k$ source or sink vertexes, $n$ of non-source and non-sink vertexes and $m$ edges in $G$, there will be $k + 2*n$ vertexes and $m + n$ edges in $G'$.

\subsection{Example}
\begin{figure}[h]
    \centering
    \includegraphics[width=0.5\linewidth]{26.1.7-1.png}
    \caption{26.1.7-1 $G$}
    \includegraphics[width=0.5\linewidth]{26.1.7-2.png}
    \caption{26.1.7-2 $G'$}
    \label{fig:enter-label}

\end{figure}


\section{26.2-2}
\begin{figure}[h]
    \centering
    \includegraphics[width=0.5\linewidth]{截屏2024-11-18 下午8.46.37.png}
    \caption{26.1(b)}
    \label{fig:enter-label}

\end{figure}
Calculated in accordance with CLRS formulas 26.9 and 26.10.

\[f(S, T) = \sum_{u \in S} \sum_{v \in T} f(u, v) - \sum_{u \in S} \sum_{v \in T} f(v, u)\]
\[c(S, T) = \sum_{u \in S} \sum_{v \in T} c(u, v)\]

\[f(S, T) = f(s,v1)+f(v2,v1)+f(v4,v3)+f(v4,t) - f(v3,v2) = 11+1+7+4-4=19\]
\[c(S, T) = c(s,v1)+c(v2,v1)+c(v4,v3)+c(v4,t)=16+4+7+4=31\]
\section{26.2-3}
\begin{figure}[h]
    \centering
    \includegraphics[width=0.5\linewidth]{截屏2024-11-18 下午8.43.56.png}
    \caption{26.1(a)}
    \label{fig:26.1(a)}
\end{figure}
By executing the Edmonds-Karp algorithm in 26.1(a), the first shortest path found is 
\[s \rightarrow v_1 \rightarrow v_3 \rightarrow t\]
The maximum flow on this path is 12.
Then updated the residual network:
\[s \rightarrow v1: 4, \quad v1 \rightarrow v3: 0, \quad v3 \rightarrow t: 8\]
\[v1 \rightarrow s: 12, \quad v3 \rightarrow v1: 12, \quad t \rightarrow v3: 12\]
Then use the updated residual network to find the next path:
\[s  \rightarrow v_2 \rightarrow v_4 \rightarrow t\]
The maximum flow on this path is 4.
Then updated the residual network:
\[s \rightarrow v2: 9, \quad v2 \rightarrow v4: 10, \quad v4 \rightarrow t: 0 \]
\[v2 \rightarrow s: 4, \quad v4 \rightarrow v2: 4, \quad t \rightarrow v4: 4 \]
Repeatedly, use the updated residual network to find the next path:
\[s \rightarrow v_2 \rightarrow v_4  \rightarrow v_3 \rightarrow t\]
The maximum flow on this path is 7.
Then updated the residual network:
\[s \rightarrow v2: 2, \quad v2 \rightarrow v4: 3, \quad v4 \rightarrow v3: 0 \quad v3 \rightarrow t: 1\]
\[v2 \rightarrow s: 11, \quad v4 \rightarrow v2: 11, \quad v3 \rightarrow v4: 7 \quad t \rightarrow v3: 19\]
At this point no other path can be found, so the total maximum flow is \(12+4+7=23\).
\section{26.2-4}
%jiayi
\textbf{Question:} In the example of Figure 26.6, what is the minimum cut corresponding to the maximum flow shown? Of the augmenting paths appearing in the example, which one cancels flow?

\textbf{Answer:} 
In Figure 26.6, the minimum cut corresponding to the maximum flow is: $S = \{s, v1, v2, v4\}$ and $T = \{v3, t\}$

\begin{equation}
c(S, T) = 19 = |f|_{max}
\end{equation}

The augmenting paths in Figure 26.6 c (s - v1 - v2 - v3 - t) cancels flow on edge (v3, v2) and get a bigger flow on edge (v3, t).

%zhigao
\section{26.2-7}
Firstly, I need to check the capacity constraint.\\
We know that : \(c_f(p) = \min \{ c_f(u, v) : (u, v) \text{ is on } p \}\)
Hence we have the following:
\[c_f(p) \leq c_f(u, v) \quad \forall (u, v) \in p.\]
So the capacity constraint is satisfied.\\
Then check the flow conservation.\\
We know the \(p\) is an augmenting path, And from the definition:
\[f_p(u, v) =
\begin{cases} 
c_f(p), & \text{if } (u, v) \text{ is on } p, \\ 
0, & \text{otherwise}.
\end{cases}\]
Assume that \(w\) is an output point and that \(w \in p\), Hence:
\[f_p(v, w) = c_f(p)\]
So we get: 
\[\sum_{u \in p} f_p(u, v) =\sum_{w \in p} f_p(v, w)\]
The flow conservation is satisfied.\\
By the definition of the augmenting path, the maximum amount by which we can increase the flow on each edge in an augmenting path is the residual capacity of \(p\).
So the flow on each edge is \(c_f(p)\), so \( |f_p| = c_f(p)> 0\)
%zhigao
\section{26.2-9}
\subsection{Flow conservation property}
The flow conservation property states that for every vertex $v$ in the network $G$ (except for the source $s$ and sink $t$), the flow entering $v$ equals the flow leaving $v$. Mathematically, for a flow $f$:
\[
\sum_{u} f(u, v) = \sum_{w} f(v, w) \quad \text{for all } v \neq s, t.
\]

When compute the augmented flow $f \uparrow f'$, the new flow at any edge $(u, v)$ is defined as:
\[
(f \uparrow f')(u, v) = f(u, v) + f'(u, v).
\]

Since both $f$ and $f'$ individually satisfy the flow conservation property:
\[
\sum_{u} f(u, v) = \sum_{w} f(v, w) \quad \text{and} \quad \sum_{u} f'(u, v) = \sum_{w} f'(v, w),
\]

Adding these equations gives:
\[
\sum_{u} \big(f(u, v) + f'(u, v)\big) = \sum_{w} \big(f(v, w) + f'(v, w)\big).
\]
This shows that $f \uparrow f'$ satisfies the flow conservation property.

\subsection{Capacity constraint}
The capacity constraint states that for every edge $(u, v)$ in the network, the flow on the edge must not exceed the edge's capacity $c(u, v)$:
\[
f(u, v) \leq c(u, v) \quad \text{for all } (u, v) \in G.
\]

For $f \uparrow f'$, the flow on any edge $(u, v)$ becomes:
\[
(f \uparrow f')(u, v) = f(u, v) + f'(u, v).
\]

If $f(u, v)$ and $f'(u, v)$ both individually satisfy the capacity constraint:
\[
f(u, v) \leq c(u, v) \quad \text{and} \quad f'(u, v) \leq c(u, v),
\]

Then their sum might exceed the capacity:
\[
f(u, v) + f'(u, v) \leq 2 \cdot c(u, v),
\]
This violates the capacity constraint unless special conditions are imposed on $f$ and $f'$.
\section{26.3-2}
\textbf{1. Initialization}  

Initially, the flow \( f \) is set to zero:
\[
f(u, v) = 0 \quad \text{for all edges } (u, v) \in G.
\]

Therefore, \( f(u, v) \) is integral, and the total flow \( |f| = 0 \) is an integer.\\
\textbf{2. Inductive argument for flow updates}  

During each iteration of the Ford-Fulkerson algorithm:
\begin{enumerate}
    \item An augmenting path \( P \) is identified in the residual graph.
    \item The bottleneck capacity \( \Delta \) of the path is computed as:
    \[
    \Delta = \min \{ c_f(u, v) \mid (u, v) \text{ is an edge in } P \},
    \]
    where \( c_f(u, v) \) is the residual capacity of edge \( (u, v) \).  
    Since \( c \) is integral and \( c_f(u, v) \) is derived from \( c \), the bottleneck \( \Delta \) is also integral.
    \item The flow along \( P \) is updated by adding \( \Delta \) to forward edges and subtracting \( \Delta \) from backward edges in \( P \). For each edge \( (u, v) \) in \( P \):
    \[
    f(u, v) \gets f(u, v) + \Delta \quad \text{(if \( (u, v) \) is a forward edge)},
    \]
    \[
    f(v, u) \gets f(v, u) - \Delta \quad \text{(if \( (u, v) \) is a backward edge)}.
    \]
\end{enumerate}

Since both the initial flow \( f \) and the update \( \Delta \) are integers, the updated flow \( f(u, v) \) remains integral.\\
\textbf{3. Termination}  

The Ford-Fulkerson algorithm terminates when no augmenting path exists in the residual graph. At this point, the flow \( f \) is a \textit{maximum flow}. Throughout the algorithm, \( f(u, v) \) remains integral because each update is based on an integral \( \Delta \).\\
\textbf{4. Total flow \( |f| \)}  

The total flow \( |f| \) is defined as the sum of flows out of the source \( s \):
\[
|f| = \sum_{v} f(s, v).
\]

Since \( f(s, v) \) is an integer for each \( v \), \( |f| \) is also an integer.
\end{document}
