\documentclass[12pt]{article}
\usepackage[utf8]{inputenc}
\usepackage{upquote}
\usepackage[margin=1in]{geometry} 
\usepackage{amsmath,amsthm,amssymb}
\usepackage{graphicx}
\usepackage{listings}
\newenvironment{statement}[2][Statement]{\begin{trivlist}
\item[\hskip \labelsep {\bfseries #1}\hskip \labelsep {\bfseries #2.}]}{\end{trivlist}}
\usepackage{xcolor}




\title{Assignment 2}


\author{Author \\
  Wanjing Hu / fng685@alumni.ku.dk  \\
  Zhigao Yan / sxd343@alumni.ku.dk  \\
  Wenshuo Dong / gnj461@alumni.ku.dk  \\
  Jiayi Zhang / xrw579@alumni.ku.dk \\
} 
 

\begin{document}
\maketitle

\section{29.1-9}
% wenshuo 
\[
\begin{aligned}
maximize& \\
& z = 3x_1 + 2x_2 \\
subject \, to&  \\
& x_1 - x_2 \leq 66, \\ 
& x_1 + x_2 \geq 77,\\
& x_1, x_2 \geq 0.
\end{aligned}
\]

The feasible region is unbounded because there is no constraint explicitly limiting \(x_1\) and \(x_2\) from growing indefinitely in certain directions.
For example, \(x_1 \to \infty\) and \(x_2 \to \infty\) is possible as long as the constraints are satisfied.

Despite the feasible region being unbounded, the objective function \(z = 3x_1 + 2x_2\) does not increase indefinitely. The key limiting constraint is \(x_1 - x_2 \leq 66\), which ensures that \(x_1\) and \(x_2\) cannot grow disproportionately. The optimal value occurs where the constraints intersect.
\section{29.2-3}
% jiayi

\section{29.2-6}
% wanjing
\textbf{Question:}
Write a linear program that, given a bipartite graph $G = (V,E)$, solves the maximum-bipartite matching problem.

\textbf{Answer}

Let $V = L \cup R$, $L \cap R = \emptyset$ . Then make a new vertex for the source which has an edge of unit capacity going to each of the vertices in $L$, and a new vertex for sink which has an edge from each of the vertices in $R$, and we have a new graph $G'=(V',E')$, where $V' = V \cup \{s,t\}$, $E' = E \cup \{s, t\}$. The capacity of edges going out of $s$ and going into $t$ are all 1. Also every original edge from left to right are edges with capacity 1.

First we need to prove the finding maximum-bipartite matching $M$ in the original bipartite graph $G$ is equivalent to finding the max flow in the graph $G$. The proof is as follows:

We first show that a matching $M$ in $G$ corresponds to an integer valued flow $f$ in $G'$. Define $f$ as follows: If $(u,v) \in M$, then $f(s,u) = f(u,v) = f(v,t) = 1$. For all other edges $(u,v) \in E'$, we define $f(u,v) = 0$. Then we have $f$ satisfying both the capacity constraint and the flow conservation.

Next, we can see each edge $(u,v) \in M$ corresponse to one unit flow in $G'$ that traverses the path $s \rightarrow u \rightarrow v \rightarrow t$. Moreover, the paths induced by edges in $M$ are vertext-disjoint, except for $s$ and $t$. The net flow across cut $(L \cup \{s\}, R \cup \{t\})$ is equal to $|M|$. thus the value of the flow $|f| = M$. 

Then we also need to prove $|M| = |f|$ in the converse side. Let $f$ be an integer-valued flow in $G'$ and let $M = \{(u,v): u \in L,\, v \in R,\, and\, f(u,v) >0\}$.

By flow conservation, each $u \in L$ has at most one unit of flow entering the edge $(s,u)$, and since $f$ is integer valued, for each $u \in L$, the one unit of flow can enter on at most one edge and can leave on at most one edge. Then one unit of flpw enters $u$ if and only if there is exactly one vertex $v \in R$ such that $f(u,v) = 1$, and at most one edge leaving each $u \in L$ carries positive flow. A symmetric proof can be made for each $v \in R$ Thus the set M is a matching.

Then the linear programming problem could be defined as follows:

\begin{equation}
\begin{aligned}
maximize& \\
& \sum_{v \in L} f_{(s,v)}\\
subject \, to&  \\
& f_{(u,v)} <=1 \, for \, each \, u,v \in \{s\} \cup L \cup R \cup \{t\} = V\\ 
& \sum_{v \in V} f_{(v,u)} = \sum_{v \in V} f_{(u,v)} \, for \, each \, u \in L \cup R\\
&f_{(u,v)} >= 0 \, for \,each \, u,v \in V
\end{aligned}
\end{equation}


\section{29.4-3}
% zhigao
\textbf{Question:}
Write down the dual of the maximum-flow linear program, as given in lines
(29.47)–(29.50) on page 860. Explain how to interpret this formulation as a
minimum-cut problem.\\
\textbf{Answer:}
Firstly, I need to convert the original formulas to standard form.
We can set a coefficient C that is 1 when \(u=s\) and -1 when \(v=s\).
\[
\begin{aligned}
maximize& \\
&\sum_{u,v \in V} Cfuv, \\
subject \, to& \\
&f_{uv} \leq c(u, v), \quad \text{for each } u, v \in V, \\
&\sum_{v \in V} f_{vu} = \sum_{v \in V} f_{uv}, \quad \text{for each } u \in V \setminus \{s, t\}, \\
&f_{uv} \geq 0, \quad \text{for each } u, v \in V.
\end{aligned}
\]
The second formula can be changed to 
\[
\sum_{v \in V} f_{vu} - \sum_{v \in V} f_{uv} \leq 0 \quad \text{for each } u \in V \setminus \{s, t\}
\]
\[-\sum_{v \in V} f_{vu} + \sum_{v \in V} f_{uv} \leq 0\ \quad \text{for each } u \in V \setminus \{s, t\}\]
Now we can get the dual subject function by the subject function, for capacity constraints and flow conservation, \(y_{uv}\), \(y_u\)(flow conservation for \(u\)) and \(y_v\)(flow conservation for \(v\))can be introduced, 

\[y_{uv}-y_u+y_v \geq C \quad \text{for each } u, v \in V\]
We can split this equation into:
\[y_{uv}-y_u+y_v \geq 0 \quad \text{for each } u, v \in V\setminus \{s, t\}\]
\[y_{sv}+y_v \geq 1 \quad \text{for each } v \in V\setminus \{s\}\]
\[y_{us}-y_u \geq -1 \quad \text{for each } u \in V\setminus \{s\}\]
\[y_{uv} \geq 0, \quad y_u \geq 0,\quad y_v\geq 0\]
The final object function is :
\[minimize \quad \sum_{u,v \in V} c(u,v)y_{uv}\]
From the object function, we can set \(y_{uv}\) as the variable for every edge, if one edge has been "cut", we need to add the capacity to the result. We can also think of \(y_u\) as the attribution of each point(without s and t). These constraints are to ensure that if two points belong to different sets, then edge \(uv\) must be cut.




\section{Linear Programming - B}
\textbf{Question:}
Write an LP in standard form in the plane so that: (1) the feasible region is a convex polygon with 5 edges, (2) the maximum of the LP is 1 and (3) the maximum is achieved on a full edge of the feasible region (and not just a single vertex)
% wenshuo

\[
\begin{aligned}
Maximize& \\
&z = x_1 + x_2, \\
Subject \, to& \\
& \quad x_1 \geq 0, \\
& \quad x_2 \geq 0, \\
& \quad x_1 + 2x_2 \leq 2, \\
& \quad 2x_1 + x_2 \leq 2, \\
& \quad x_1 - x_2 \leq 1.
\end{aligned}
\]


The constraints define a convex polygon in the \(x_1\)-\(x_2\) plane. Each inequality corresponds to a half-space, and their intersection forms the feasible region. Since there are 5 constraints, the feasible region is bounded and consists of 5 edges.

The objective function \(z = x_1 + x_2\) increases as \(x_1\) and \(x_2\) increase. The maximum value of \(z\) within the feasible region occurs where \(x_1 + x_2 = 1\), which is entirely contained within the feasible region. Thus, the maximum value is \(z = 1\).

The edge of the feasible region defined by \(x_1 + x_2 = 1\) (subject to the constraints) is part of the feasible region. This ensures that the maximum value of \(z = 1\) is achieved along this edge, not just at a single vertex.

The feasible region can be visualized as a convex polygon with vertices formed by solving the intersection of pairs of constraints. The line \(x_1 + x_2 = 1\) intersects this polygon along an entire edge where the maximum is achieved.

\section{Linear Programming - C}
\textbf{Question}
This exercise demonstrates that LP can represent non-linearities (like absolute values). Let f be a function from the set {1,2} to the real numbers. 

Write an LP whose value is $|f(1)|+|f(2)|$, and write the dual to this LP. In your solutions, all the entries in the matrix A and in the vectors b,c must be affine functions in $f(1)$ and $f(2)$; that is, each entry must be of the form $d f(1) + e f(2) + g$ where $d$,$e$,$g$ are real numbers. For example, in dimension n = 3, the vector b can be $(3 f(1), 0, -2f(2)+1)$ but not $(|f(1)|,0,0)$. 
% jiayi


\section{Randomized Algorithm-1}
\textbf{Question}
For any given key $x$ in $S$, let $d(x)$ be the depth of $x$ in the search tree generated by $RandQS$. Give a lower and an upper bound, as a function of the number of keys n, on the expected value of $d(x)$ such that these bounds are within a constant factor from each other. That is, find a function $f(n)$ and prove that $E[d(x)]=\Theta(f(n))$.

% wanjing
\textbf{Answer}
Let $S$ be the set of the $n$ keys, then for $x,y \in S$ the comparison between $x$ and $y$ will occur when $x$ and $y$ are in the same subproblem during the step of a recursion, and one of them is chosen as a pivot. For any pair $(x,y)$, we note the probability that $x$, $y$ are in the same subproblem as $A$. Then we have:

\begin{equation}
P(A) = 2/n
\end{equation}
Also, from the definition of $d(x)$ we can see $d(x)$ equals to the number of comparisons involving $x$ during the partition process. Then we have 

\begin{equation}
\begin{aligned}
E(d(x)) &= \sum_{k=1}^{n-1} 2/k\\
&= 2*(1 + 1/2 + 1/3 + ... + 1/n-1)
\end{aligned}
\end{equation}

Since we have:
\begin{equation}
\begin{aligned}
\int_{1}^{n-1} \frac{1}{x} dx = log(n-1)
\end{aligned}
\end{equation}

So we have
\begin{equation}
\begin{aligned}
E[d(x)] &<= 2*(1 + log(n-1))\\
E[d(x)] &>= 2(*log(n-1))
\end{aligned}
\end{equation}

That is:
\begin{equation}
\begin{aligned}
E[d(x)] &<= 2*log(n) + C\\
E[d(x)] &>= 2*log(n) - C
\end{aligned}
\end{equation}

So we have
\begin{equation}
\begin{aligned}
E[d(x)] = \Theta(log(n))\\
\end{aligned}
\end{equation}


\section{Randomized Algorithm-2}
\textbf{Question}
How many runs of randomized contraction do we need to be 99\% sure that a minimum cut is found?
% zhigao
\textbf{Answer}:
Every time we run the randomized contraction, the probability of min-cut is returned is (n is the number of vectors in the graph):
\[P\geq\frac{2}{n(n-1)}\]
So the probability of min-cut is not returned is :
\[P\geq1-\frac{2}{n(n-1)}\]
By t times, this probability is :
\[(1-\frac{2}{n(n-1)})^t\]
We want to ensure 99\% that a minimum cut is found, Hence:
\[(1-\frac{2}{n(n-1)})^t\leq0.01\]
\[=tln(1-\frac{2}{n(n-1)})\leq ln(0.01)\]
\[=t\leq \frac{ln(0.01)}{ln(1-\frac{2}{n(n-1)})}\]
By using the \(1+x\leq e^x \rightarrow ln(1+x)\leq x\), we get:
\[=t\leq \frac{ln(0.01)}{-\frac{2}{n(n-1)}}\]
Finally, If we want to guarantee more than 99\% probability, then the number of times \(t\) needs to satisfy the following condition.
\[
t \geq \frac{4.605 \cdot n(n - 1)}{2}.
\]


\section{Randomized Algorithm-exercises 1.2}
% wenshuo
\begin{enumerate}
    \item Let \( G = (V, E) \) be the graph with \( n \) vertices and \( m \) edges. The minimum cut \( C_{\text{min}} \) is the smallest cut, i.e., the fewest edges that separate two disconnected sets of vertices.
    
    \item Instead of contracting an edge, we randomly select two vertices and merge them into a single vertex. This operation does not necessarily preserve the cut structure in the same way that edge contraction does.

    \item In Karger's algorithm, each edge contraction preserves the structure of the minimum cut, while the modified algorithm may coalesce vertices in a way that disrupts the cut structure. Specifically, if two vertices from different parts of the min-cut are merged, the cut structure is altered, making it more difficult to find the actual minimum cut in future steps.

    \item Suppose the minimum cut consists of \( k \) edges separating two distinct sets of vertices. Each time we randomly choose two vertices to merge, there is a probability that we merge vertices from different sets of the cut, which disturbs the min-cut structure.
    
    The probability that we make an appropriate choice that preserves the minimum cut structure decreases with each step. After a number of coalescences, the algorithm may have merged vertices in such a way that the min-cut is no longer detectable.

    \item In Karger's algorithm, the probability of failure after \( O(n^2) \) edge contractions is small, specifically \( O(1/n^2) \). However, in the modified algorithm, each coalescence step is less efficient because it does not directly focus on the cut structure. As a result, the number of steps required to find a valid min-cut may increase, and the probability of success becomes exponentially small in \( n \).

    \item After \( O(n^2) \) steps, the probability of finding the min-cut may be exponentially small. The reason is that each random selection of two vertices has a probability of failing to preserve the cut, and this failure accumulates exponentially over time.
\end{enumerate}



\section{Randomized Algorithm-exercises 1.3}
%you may assume that T(n) is the worst case running time of A.
% jiayi

\end{document}
