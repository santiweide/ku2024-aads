\documentclass[12pt]{article}
\usepackage[utf8]{inputenc}
\usepackage{upquote}
\usepackage[margin=1in]{geometry} 
\usepackage{amsmath,amsthm,amssymb}
\usepackage{graphicx}
\usepackage{listings}
\newenvironment{statement}[2][Statement]{\begin{trivlist}
\item[\hskip \labelsep {\bfseries #1}\hskip \labelsep {\bfseries #2.}]}{\end{trivlist}}
\usepackage{xcolor}




\title{Assignment 1}


\author{Author \\
  Wanjing Hu / fng685@alumni.ku.dk  \\
  Zhigao Yan / sxd343@alumni.ku.dk  \\
  Wenshuo Dong / gnj461@alumni.ku.dk  \\
  Jiayi Zhang / xrw579@alumni.ku.dk \\
} 
 

\begin{document}
\maketitle

\section{29.1-9}
% wenshuo 

\section{29.2-3}
% jiayi

\section{29.2-6}
% wanjing
\textbf{Question:}
Write a linear program that, given a bipartite graph $G = (V,E)$, solves the maximum-bipartite matching problem.

\textbf{Answer}

Let $V = L \cup R$, $L \cap R = \emptyset$ . Then make a new vertex for the source which has an edge of unit capacity going to each of the vertices in $L$, and a new vertex for sink which has an edge from each of the vertices in $R$, and we have a new graph $G'=(V',E')$, where $V' = V \cup \{s,t\}$, $E' = E \cup \{s, t\}$. The capacity of edges going out of $s$ and going into $t$ are all 1. Also every original edge from left to right are edges with capacity 1.

First we need to prove the finding maximum-bipartite matching $M$ in the original bipartite graph $G$ is equivalent to finding the max flow in the graph $G$. The proof is as follows:

We first show that a matching $M$ in $G$ corresponds to an integer valued flow $f$ in $G'$. Define $f$ as follows: If $(u,v) \in M$, then $f(s,u) = f(u,v) = f(v,t) = 1$. For all other edges $(u,v) \in E'$, we define $f(u,v) = 0$. Then we have $f$ satisfying both the capacity constraint and the flow conservation.

Next, we can see each edge $(u,v) \in M$ corresponse to one unit flow in $G'$ that traverses the path $s \rightarrow u \rightarrow v \rightarrow t$. Moreover, the paths induced by edges in $M$ are vertext-disjoint, except for $s$ and $t$. The net flow across cut $(L \cup \{s\}, R \cup \{t\})$ is equal to $|M|$. thus the value of the flow $|f| = M$. 

Then we also need to prove $|M| = |f|$ in the converse side. Let $f$ be an integer-valued flow in $G'$ and let $M = \{(u,v): u \in L,\, v \in R,\, and\, f(u,v) >0\}$.

By flow conservation, each $u \in L$ has at most one unit of flow entering the edge $(s,u)$, and since $f$ is integer valued, for each $u \in L$, the one unit of flow can enter on at most one edge and can leave on at most one edge. Then one unit of flpw enters $u$ if and only if there is exactly one vertex $v \in R$ such that $f(u,v) = 1$, and at most one edge leaving each $u \in L$ carries positive flow. A symmetric proof can be made for each $v \in R$ Thus the set M is a matching.

Then the linear programming problem could be defined as follows:

\begin{equation}
\begin{aligned}
maximize& \\
& \sum_{v \in L} f_{(s,v)}\\
subject \, to&  \\
& f_{(u,v)} <=1 \, for \, each \, u,v \in \{s\} \cup L \cup R \cup \{t\} = V\\
& \sum_{v \in V} f_{(v,u)} = \sum_{v \in V} f_{(u,v)} \, for \, each \, u \in L \cup R\\
&f_{(u,v)} >= 0 \, for \,each \, u,v \in V
\end{aligned}
\end{equation}


\section{29.4-3}
% zhigao

\section{Linear Programming - B}
\textbf{Question:}
Write an LP in standard form in the plane so that: (1) the feasible region is a convex polygon with 5 edges, (2) the maximum of the LP is 1 and (3) the maximum is achieved on a full edge of the feasible region (and not just a single vertex)
% wenshuo

\section{Linear Programming - C}
\textbf{Question}
This exercise demonstrates that LP can represent non-linearities (like absolute values). Let f be a function from the set {1,2} to the real numbers. 

Write an LP whose value is $|f(1)|+|f(2)|$, and write the dual to this LP. In your solutions, all the entries in the matrix A and in the vectors b,c must be affine functions in $f(1)$ and $f(2)$; that is, each entry must be of the form $d f(1) + e f(2) + g$ where $d$,$e$,$g$ are real numbers. For example, in dimension n = 3, the vector b can be $(3 f(1), 0, -2f(2)+1)$ but not $(|f(1)|,0,0)$. 
% jiayi


\section{Randomized Algorithm-1}
\textbf{Question}
For any given key x in S, let d(x) be the depth of x in the search tree generated by RandQS. Give a lower and an upper bound, as a function of the number of keys n, on the expected value of d(x) such that these bounds are within a constant factor from each other. That is, find a function f(n) and prove that E[d(x)]=Theta(f(n)).
% wanjing
\textbf{Answer}


\section{Randomized Algorithm-2}
\textbf{Question}
How many runs of randomized contraction do we need to be 99\% sure that a minimum cut is found?
% zhigao

\section{Randomized Algorithm-exercises 1.2}
% wenshuo

\section{Randomized Algorithm-exercises 1.3}
%you may assume that T(n) is the worst case running time of A.
% jiayi

\end{document}
