\documentclass[12pt]{article}
\usepackage[utf8]{inputenc}
\usepackage{upquote}
\usepackage[margin=1in]{geometry} 
\usepackage{amsmath,amsthm,amssymb}
\usepackage{graphicx}
\usepackage{listings}
\newenvironment{statement}[2][Statement]{\begin{trivlist}
\item[\hskip \labelsep {\bfseries #1}\hskip \labelsep {\bfseries #2.}]}{\end{trivlist}}
\usepackage{xcolor}




\title{Assignment 1}


\author{Author \\
  Wanjing Hu / fng685@alumni.ku.dk  \\
  Zhigao Yan / sxd343@alumni.ku.dk  \\
  Wenshuo Dong / gnj461@alumni.ku.dk  \\
  Jiayi Zhang / xrw579@alumni.ku.dk \\
} 
 

\begin{document}
\maketitle

\section{29.1-9}
% wenshuo 

\section{29.2-3}
% jiayi

\section{29.2-6}
% wanjing


\section{29.4-3}
% zhigao
\textbf{Question:}
Write down the dual of the maximum-flow linear program, as given in lines
(29.47)–(29.50) on page 860. Explain how to interpret this formulation as a
minimum-cut problem.\\
\textbf{Answer:}

\section{Linear Programming - B}
\textbf{Question:}
Write an LP in standard form in the plane so that: (1) the feasible region is a convex polygon with 5 edges, (2) the maximum of the LP is 1 and (3) the maximum is achieved on a full edge of the feasible region (and not just a single vertex)
% wenshuo

\section{Linear Programming - C}
\textbf{Question}
This exercise demonstrates that LP can represent non-linearities (like absolute values). Let f be a function from the set {1,2} to the real numbers. 

Write an LP whose value is $|f(1)|+|f(2)|$, and write the dual to this LP. In your solutions, all the entries in the matrix A and in the vectors b,c must be affine functions in $f(1)$ and $f(2)$; that is, each entry must be of the form $d f(1) + e f(2) + g$ where $d$,$e$,$g$ are real numbers. For example, in dimension n = 3, the vector b can be $(3 f(1), 0, -2f(2)+1)$ but not $(|f(1)|,0,0)$. 
% jiayi


\section{Randomized Algorithm-1}
\textbf{Question}
For any given key x in S, let d(x) be the depth of x in the search tree generated by RandQS. Give a lower and an upper bound, as a function of the number of keys n, on the expected value of d(x) such that these bounds are within a constant factor from each other. That is, find a function f(n) and prove that E[d(x)]=Theta(f(n)).
% wanjing

\section{Randomized Algorithm-2}
\textbf{Question}
How many runs of randomized contraction do we need to be 99\% sure that a minimum cut is found?
% zhigao

\section{Randomized Algorithm-exercises 1.2}
% wenshuo

\section{Randomized Algorithm-exercises 1.3}
%you may assume that T(n) is the worst case running time of A.
% jiayi

\end{document}
